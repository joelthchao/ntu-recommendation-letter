\documentclass[12pt,letterpaper]{article}

\NeedsTeXFormat{LaTeX2e}
\usepackage{enumitem}
\usepackage[left=1in,top=0.5in,right=1in,bottom=0.8in]{geometry}
\usepackage{graphicx}
\usepackage{setspace}

\RequirePackage{xeCJK}
 
\setmainfont{Times New Roman}
\setCJKmainfont[AutoFakeBold=true]{標楷體}

\thispagestyle{empty}
\setlength{\parindent}{0em}
\setlength{\parskip}{1em}

\begin{document}

\begin{center}
\begin{minipage}{0.12\textwidth}
\includegraphics[width=\linewidth]{ntu_logo}
\end{minipage}
\hspace{1em}
\begin{minipage}{0.55\textwidth}
\textbf{國立臺灣大學資訊網路與多媒體研究所}\\
\textbf{\textit{Graduate Institute of Networking and Multimedia}}\\
\textbf{National Taiwan University}\\
\vspace{-1.2em}
\begin{spacing}{0.75}
{\footnotesize
台北市 106 羅斯福路四段一號 資訊工程德田館 218 室\\
No. 1, Roosevelt Rd. Sec. 4, Taipei, 106, TAIWAN\\
TEL:+886-2-33664899 FAX:+886-2-33664898
http://www.inm.ntu.edu.tw/
}
\end{spacing}
\end{minipage}
\\ \vspace{-5pt}\linethickness{0.5mm}\line(1,0){470}
\end{center}

\begin{spacing}{1.05}
Dear Committee Member:

I am sending this letter to recommend Ting-Hsuan Chao for student travel grant for ACM Multimedia 2015. He is the 2nd-year Master student in National Taiwan University under my supervision. He had been active and devoted to advanced research for multimedia content analysis. Chao, as the CO-AUTHOR, has the ACM Multimedia 2015 Grand Challenge paper accepted for the conference. In this work, titled \textit{“Unsupervised Latent Sub-events Discovery based on Multi-content and Human Activities Analysis for Diverse Event Summarization”}, we seek to discover sub-events with diversity. He is also a co-author of a short paper, titled \textit{"Filter-Invariant Image Classification on Social Media Photos"}, proposing a novel CNN architecture that utilizes the power of pairwise constraint by combining Siamese network and the adaptive margin contrastive loss with discriminative pair sampling method to solve the problem of filter bias.

He had shown his talent and enthusiasm for advanced research. He had published a high-quality top conference papers in IEEE CVPR 2015 as the first author. He had demonstrated his capability in advanced research. He enjoyed addressing the challenging problems.

ACM Multimedia is the best conference in multimedia/signal community. The grant will provide him the great opportunity for closed interactions with the first-tier research groups and for bringing him to the state-of-the-art research works presented in the venue. We do believe that we are to educate a passionate and promising researcher in multimedia research. Please consider him excellence for the travel grant and contact me should you need further information.

\vfill
Best Regards,

\end{spacing}

\begin{tabbing}
XXXXXXX XXX (某某某)\\
Professor, \= Department of Computer Science and Information Engineering and \\
\> Graduate Institute of Network and Multimedia\\
National Taiwan University\\
xxx@ntu.edu.tw; TEL: +886-2-xxxxxxxx ext. xxx
\end{tabbing}



\end{document}
